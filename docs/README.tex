\documentclass{article}
\usepackage{amsmath}
\usepackage{amssymb}
\usepackage{listings}
\usepackage{hyperref}
\usepackage{xcolor}

\definecolor{codegray}{rgb}{0.5,0.5,0.5}
\definecolor{codeblue}{rgb}{0.36,0.54,0.66}
\definecolor{backcolor}{rgb}{0.95,0.95,0.92}

\lstdefinestyle{mystyle}{
backgroundcolor=\color{backcolor},
commentstyle=\color{codegray},
keywordstyle=\color{codeblue},
numberstyle=\tiny\color{codegray},
stringstyle=\color{codegray},
basicstyle=\ttfamily\footnotesize,
breaklines=true,
captionpos=b,
keepspaces=true,
numbers=left,
numbersep=5pt,
showspaces=false,
showstringspaces=false,
showtabs=false,
tabsize=2
}

\lstset{style=mystyle}


\title{Labwork 3. LaTeX}
\author{Vatlasov Savely Andreevich}
\date{}

\begin{document}

\maketitle

\begin{center}
    \begin{tabular}{ll}
    \textbf{Reviewer: Schuikov Artem Sergeevich} & \\ 
    \textbf{Group: M3110} & \\ 
    \textbf{Github for review: \href{https://github.com/llemonthefrog/geometric_lib}{https://github.com/llemonthefrog/geometric_lib}}
    \end{tabular}
\end{center}

\newpage
\tableofcontents

\subsection*{1. About the Project}
\subsection*{2. How to Use Calculator}
\subsection*{3. Math Formulas}
    a. Area \\
    b. Perimeter \\
\subsection*{4. Functions}
    a. Rectangles \\
    b. Circle \\
    c. Triangle \\
    d. Square \\
    e. Calculate \\

\newpage


\section*{About the Project}

The project is a calculator for the area and perimeter of various geometric shapes. It supports the following shapes:
\begin{itemize}
\item Circles
\item Squares
\item Rectangles
\item Triangles
\end{itemize}

The user can select the shape and type of operation (area or perimeter) to perform calculations. Functions are implemented in accordance with their mathematical formulas.

\section*{How to Use Calculator}

\begin{enumerate}
\item Run \texttt{python calculate.py}
\item Enter the figure name. Available are Circle, Square.
\item Enter the function: Area or Perimeter.
\item Enter figure sizes. Radius for circle, one side for square.
\item Get the answer!
\end{enumerate}

\section*{Math Formulas}

\subsection*{Area}
\begin{itemize}
\item Circle: \( S = \pi R^2 \)
\item Rectangle: \( S = ab \)
\item Square: \( S = a^2 \)
\item Triangle: \( S = \sqrt{p \cdot (p-a) \cdot (p-b) \cdot (p-c)} \) where \( p \) is the semiperimeter
\end{itemize}

\subsection*{Perimeter}
\begin{itemize}
\item Circle: \( P = 2\pi R \)
\item Rectangle: \( P = 2a + 2b \)
\item Square: \( P = 4a \)
\item Triangle: \( P = a + b + c \)
\end{itemize}

\section*{Functions}

\subsection*{Rectangles}

Finds the area using the formula for the area of ​​a rectangle
\begin{lstlisting}[language=Python, caption={Area Function for Rectangle}]
def area(a, b):
return a * b

# Examples:
area(10, 20) # 200

area(1, 2) # 2
\end{lstlisting}

Finds the perimeter using the formula for the perimeter of ​​a rectangle
\begin{lstlisting}[language=Python, caption={Perimeter Function for Rectangle}]
def perimeter(a, b):
return 2 * (a + b)

# Examples:
perimeter(10, 20) # 60

perimeter(1, 2) # 6
\end{lstlisting}

\subsection*{Circle}

Finds the area using the formula for the area of ​​a circle
\begin{lstlisting}[language=Python, caption={Area Function for Circle}]
def area(r):
return 3.14159 * (r ** 2)

# Examples:
area(10)# 314.159...

area(300)# 282743.338...
\end{lstlisting}

Finds the perimeter using the formula for the perimeter of ​​a circle
\begin{lstlisting}[language=Python, caption={Perimeter Function for Circle}]
def perimeter(r):
return 2 * 3.14159 * r

# Examples:
perimeter(10)# 62.831...

perimeter(300)# 1884.955...
\end{lstlisting}

\subsection*{Triangle}

Finds the area using the formula for the area of ​​a triangle
\begin{lstlisting}[language=Python, caption={Area Function for Triangle}]
def area(a, b, c):
p = (a + b + c) / 2
return (p * (p - a) * (p - b) * (p - c)) ** 0.5

# Examples:
area(3, 4, 5)# 6

area(7, 8, 9)# 12
\end{lstlisting}

Finds the perimeter using the formula for the perimeter of ​​a circle
\begin{lstlisting}[language=Python, caption={Perimeter Function for Triangle}]
def perimeter(a, b, c):
return a + b + c

# Examples:
perimeter(3, 4, 5)# 12

perimeter(7, 8, 9)# 24
\end{lstlisting}

\subsection*{Square}

Finds the area using the formula for the area of ​​a square
\begin{lstlisting}[language=Python, caption={Area Function for Square}]
def area(a):
return a ** 2

# Examples:
area(10)# 100

area(300)# 90000
\end{lstlisting}

Finds the perimeter using the formula for the perimeter of ​​a square
\begin{lstlisting}[language=Python, caption={Perimeter Function for Square}]
def perimeter(a):
"""Takes the value of side a."""
return 4 * a

# Examples:
# perimeter(10)
# 40

# perimeter(300)
# 1200
\end{lstlisting}

\subsection*{Calculate}

A function that calculates values for certain predefined shapes.
\begin{lstlisting}[language=Python, caption={Calculate Function}]
def calc(fig, func, size):
if fig == "circle" and func == "area":
return area(size[0])
# Continue with other conditions

# Examples:
calc("circle", "area", [5])# 78.539...

calc("rectangle", "perimeter", [5, 10])# 30
\end{lstlisting}

\end{document}